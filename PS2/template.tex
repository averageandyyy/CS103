%
% LaTeX Problem Set Template originally designed
% by former CS103 TA Sachin Padmanabhan, with updates,
% edits, and simplifications by the lovely folks below.
%
% Updated for Fall 2018 by Michael Zhu
% Updated for Fall 2019 by Joshua Spayd
% Updated for Fall 2020 by Lucy Lu
% Updated for Winter/Spring 2021 by Cynthia Bailey Lee
% Updated for Fall 2021 by Grant McClearn
% Commands slimmed down and simplified in Fall 2022 by Keith Schwarz

\documentclass{article}
\usepackage{amsmath}
\usepackage{amssymb}
\usepackage{amsthm}
\usepackage{amssymb}
\usepackage{mathdots}
\usepackage{braket}
\usepackage[pdftex]{graphicx}
\usepackage{fancyhdr}
\usepackage[margin=1in]{geometry}
\usepackage{multicol}
\usepackage{bm}
\usepackage{listings}
\PassOptionsToPackage{usenames,dvipsnames}{color}  %% Allow color names
\usepackage{pdfpages}
\usepackage{algpseudocode}
\usepackage{tikz}
\usepackage{enumitem}
\usepackage[T1]{fontenc}
\usepackage{inconsolata}
\usepackage{framed}
\usepackage{wasysym}
\usepackage[thinlines]{easytable}
\usepackage{hyperref}
\usepackage{wrapfig}
\usepackage{color, soul}
\hypersetup{
    colorlinks=true,
    linkcolor=blue,
    filecolor=magenta,
    urlcolor=blue,
}

\title{CS 103: Mathematical Foundations of Computing\\Problem Set \#2}
\author{[TODO: Replace this with your name(s)]}
\date{\today}

% Running author based on https://tex.stackexchange.com/questions/68308/how-to-add-running-title-and-author#answer-68310
\makeatletter
\let\runauthor\@author
\makeatother

\lhead{\runauthor}
\chead{Problem Set 2}
\rhead{\today}
\lfoot{}
\cfoot{CS 103: Mathematical Foundations of Computing --- Summer 2024}
\rfoot{\thepage}

\newcommand{\abs}[1]{\lvert #1 \rvert}
\renewcommand{\(}{\left(}
\renewcommand{\)}{\right)}
\newcommand{\floor}[1]{\left\lfloor#1\right\rfloor}
\newcommand{\ceil}[1]{\left\lceil#1\right\rceil}
\newcommand{\pd}[1]{\frac{\partial}{\partial #1}}
\newcommand{\powerset}[1]{\wp\left(#1\right)}
\newcommand{\suchthat}{\ \vert \ }
\newcommand{\naturals}{\mathbb{N}}
\newcommand{\integers}{\mathbb{Z}}
\newcommand{\reals}{\mathbb{R}}
\renewcommand{\qed}{\blacksquare}
\newcommand{\accepts}{\text{ accepts }}
\newcommand{\rejects}{\text{ rejects }}
\newcommand{\loopson}{\text{ loops on }}
\newcommand{\haltson}{\text{ halts on }}
\newcommand{\encoded}[1]{\left\langle#1\right\rangle}
\newcommand{\rlangs}{\mathbf{R}}
\newcommand{\relangs}{\mathbf{RE}}
\newcommand{\corelangs}{\text{co-}\mathbf{RE}}
\newcommand{\plangs}{\mathbf{P}}
\newcommand{\nplangs}{\mathbf{NP}}

\renewcommand{\labelitemii}{$\bullet$}
\renewcommand\qedsymbol{$\blacksquare$}
\newenvironment{prf}{{\bfseries Proof.}}{\qedsymbol}
\renewcommand{\emph}[1]{\textit{\textbf{#1}}}
\newcommand{\annotate}[1]{\textit{\textcolor{blue}{#1}}}
\usepackage{mdframed}
\usepackage{float}

\makeatother

\definecolor{shadecolor}{gray}{0.95}

\theoremstyle{plain}
\newtheorem*{lem}{Lemma}

\theoremstyle{plain}
\newtheorem*{claim}{Claim}

\theoremstyle{definition}
\newtheorem*{answer}{Answer}

\newtheorem{theorem}{Theorem}[section]
\newtheorem*{thm}{Theorem}
\newtheorem{corollary}{Corollary}[theorem]
\newtheorem{lemma}[theorem]{Lemma}

\renewcommand{\headrulewidth}{0.4pt}
\renewcommand{\footrulewidth}{0.4pt}

\setlength{\parindent}{0pt}

\pagestyle{fancy}

\renewcommand{\thefootnote}{\fnsymbol{footnote}}

\usepackage{boxedminipage}
\newenvironment{blank}{\colorbox{shadecolor}{\strut \underline{\ \ \ \ \ }}}

\begin{document}

\maketitle

\begin{center}
  \emph{Due Friday, July 12 at 5:30 pm Pacific}
\end{center}

\vspace{1cm}

Problems One through Six are to be answered by editing the appropriate files (see the Problem Set \#2 instructions). You won't include your answers to those problems here.

\section*{Symbols Reference}
Here are some symbols that may be useful for this PSet. If you are using \LaTeX, view this section in the template file (the code in \texttt{cs103-ps2-template.tex}, not the PDF) and copy-paste math code snippets from the list below into your responses, as needed. If you are typing your Pset in another program such as Microsoft Word, you should be able to copy some of the symbols below from this PDF and paste them into your program. Unfortunately the symbols with a slash through them (for ``not'') and font formats such as exponents don't usually copy well from PDF, but you may be able to type them in your editor using its built-in tools.
\begin{itemize}
    \item Logical AND: $\land$
    \item Logical OR: $\lor$
    \item Logical NOT: $\lnot$
    \item Logical implies: $\to$
    \item Logical biconditional: $\leftrightarrow$
    \item Logical TRUE: $\top$
    \item Logical FALSE: $\bot$
    \item Universal quantifier: $\forall$
    \item Existential quantifier: $\exists$
\end{itemize}

\LaTeX typing tips:
\begin{itemize}
    \item Set (curly braces need an escape character backslash): ${1, 2, 3}$ (incorrect), $\{1, 2, 3\}$ (correct)
    \item Exponents (use curly braces if exponent is more than 1 character): $x^2$, $2^{3x}$
    \item Subscripts (use curly braces if subscript is more than 1 character): $x_0$, $x_{10}$
\end{itemize}
    
\newpage

\section*{Problem Seven: Yablo's Paradox}
    i.
    \begin{shaded}
        \begin{itemize}
            \item 
                Theorem: There does not exist a natural number $n$ where the statement $S_{n}$ is true.
            \item 
                Proof: We wish to show that there does not a natural number $n$ where the statement $S_{n}$ is true. Assume, for the sake of contradiction, that there exists a natural number $n$ where the statement $S_{n}$ is true. 
            \item
                Since $S_{n}$ is true, we know that for any natural number $m$, such that $m>n$, the statement $S_{m}$ is false. This allows us to conclude that the statement $S_{n+1}$ is false. Since statement $S_{n+1}$ is false, we know that there exists some natural number $l$, such that $l>n+1$, where the statement $S_{l}$ is true.
            \item 
                However, this is impossible due to our prior assumption and the fact that $l>n$, which implies that the statement $S_{l}$ must be false. We have arrived at a contradiction and so our assumption must have been wrong. Therefore, there does not exist a natural number $n$ where the statement $S_{n}$ is true. $\blacksquare$
        \end{itemize}
    \end{shaded}
    
    ii.
    \begin{shaded}
        \begin{itemize}
            \item 
                Theorem: There does not exist a natural number $n$ where the statement $S_{n}$ is false.
            \item 
                Proof: We want to show that there does not exist a natural number $n$ where the statement $S_{n}$ is false. Assume for the sake of contradiction that there exists a natural number $n$ such that the statement $S_{n}$ is false.
            \item 
                Since the statement $S_{n}$ is false, we know that there must exist a natural number $m$, such that $m>n$, where the statement $S_{m}$ is true. Since $S_{m}$ is true, we know that for any natural number $l$, such that $l>m$, the statement $S_{l}$ is false. Similar the the previous question, since $S_{l}$ is false, there must exist a natural number $o$, such that $o>l$, where the statement $S_{o}$ is true. However, this is impossible due to the earlier fact that since $o>m$, the statement $S_{o}$ must be false. We have arrived at a contradiction, which means our inital assumption must have been wrong. Therefore, there does not exist a natural number $n$ where the statement $S_{n}$ is false. $\blacksquare$
        \end{itemize}
    \end{shaded}
    
    iii.
    \begin{shaded}
        \begin{itemize}
            \item
                Evaluating $T_{0}$:
                    \begin{itemize}
                        \item 
                            Since there are no statements after $T_{9,999,999,999}$, this statement is vacuosly true.
                        \item 
                            Any statement $T_{n}$, where $0<=n<9,999,999,999$, is false, since if they were true, it must mean that $T_{9,999,999,999}$ is false, which is impossible since as mentioned earlier, it is vacuosly true.
                    \end{itemize}
        \end{itemize}
    \end{shaded}

\newpage

\section*{Problem Eight: Hereditary Sets}
    i.
    \begin{shaded}
        \begin{itemize}
            \item 
                Theorem: There exists a hereditary set.
            \item 
                Proof: We wish to show that there exists a hereditary set. Let $S=\emptyset$. For $S$ to be a hereditary set, we need to show that every element in $S$ is also a hereditary set.
            \item 
                Since $S$ is the empty set and has no elements, the proposition that every element in $S$ is an hereditary set is vacuosly true. Therefore, $S$ itself is a hereditary set. $\blacksquare$
        \end{itemize}
    \end{shaded}
        
    ii.
    \begin{shaded}
        \begin{itemize}
            \item 
                Theorem: If $S$ is a hereditary set, then $\powerset{S}$ is also a hereditary set.
            \item 
                Proof: Assume that $S$ is a hereditary set. We wish to prove that $\powerset{S}$ is a hereditary set. To do so, we need to show that any $y\in \powerset{S}$ is a hereditary set.
            \item
                Since $S$ is a hereditary set, any $x\in S$ is also a hereditary set. Since any $x\in S$ is a hereditary set, any subset of $S$ is also a hereditary set. And since $\powerset{S}$ produces the set whose elements are all possible subsets of $S$, which are themselves hereditary sets, any $y\in \powerset{S}$ is also a hereditary set, which is what we wanted to show. $\blacksquare$ \hl{Check for repeating definition.}
        \end{itemize}
    \end{shaded}
    
\newpage

\section*{Problem Nine: Tournament Champions}
    i.
    \begin{shaded}
        \begin{itemize}
            \item 
                D is not a champion. E is a champion. $\blacksmiley$
        \end{itemize}
    \end{shaded}
    
    ii.
    \begin{shaded}
        \begin{itemize}
            \item 
                Theorem: If player $c$ won more games than anyone else in $T$ or is tied for winning the greatest number of games, then $c$ is a tournament champion.
            \item
                Proof: Assume for the sake of contradiction that there exists a player $c$ such that $c$ won more games or is tied for winning the greatest number of games and $c$ is not a tournament champion.
            \item 
                Since $c$ is not a champion, there must exist a player $p$ such that $p$ beat $c$ and for any other player $q$, $q$ beat $c$ or $p$ beat $q$. Assume that the tournament now only consists of the two players $p$ and $c$. Since $c$ is not a champion, $p$ beat $c$. This must mean that $p$ has more wins than $c$, which is impossible since earlier we assumed that $c$ must have either won the most games or is tied for the greatest number of wins. We have arrived at a contradiction, which means our inital assumption must have been wrong.
            \item 
                Therefore, if player $c$ won the most games or is tied for the most number of wins, then $c$ is a tournament champion. $\blacksquare$
        \end{itemize}
    \end{shaded}
    
\newpage

\section*{Optional Fun Problem: Insufficient Connectives}

\begin{shaded}
    \begin{itemize}
        \item 
            Theorem: We cannot express every possible propositional logic formula using just $\leftrightarrow$ and $\bot$.
    \end{itemize}
\end{shaded}
    
\end{document}