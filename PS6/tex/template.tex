%
% LaTeX Problem Set Template originally designed
% by former CS103 TA Sachin Padmanabhan, with updates,
% edits, and simplifications by the lovely folks below.
%
% Updated for Fall 2018 by Michael Zhu
% Updated for Fall 2019 by Joshua Spayd
% Updated for Fall 2020 by Lucy Lu
% Updated for Winter/Spring 2021 by Cynthia Bailey Lee
% Updated for Fall 2021 by Grant McClearn
% Commands slimmed down and simplified in Fall 2022 by Keith Schwarz

\documentclass{article}
\usepackage{amsmath}
\usepackage{amssymb}
\usepackage{amsthm}
\usepackage{amssymb}
\usepackage{mathdots}
\usepackage{braket}
\usepackage[pdftex]{graphicx}
\usepackage{fancyhdr}
\usepackage[margin=1in]{geometry}
\usepackage{multicol}
\usepackage{bm}
\usepackage{listings}
\PassOptionsToPackage{usenames,dvipsnames}{color}  %% Allow color names
\usepackage{pdfpages}
\usepackage{algpseudocode}
\usepackage{tikz}
\usepackage{enumitem}
\usepackage[T1]{fontenc}
\usepackage{inconsolata}
\usepackage{framed}
\usepackage{wasysym}
\usepackage[thinlines]{easytable}
\usepackage{hyperref}
\usepackage{wrapfig}
\usepackage{mathrsfs}
\hypersetup{
    colorlinks=true,
    linkcolor=blue,
    filecolor=magenta,
    urlcolor=blue,
}

\title{CS 103: Mathematical Foundations of Computing\\Problem Set \#6}
\author{Cheng Jia Wei Andy, Xiang Qiuyu}
\date{\today}

% Running author based on https://tex.stackexchange.com/questions/68308/how-to-add-running-title-and-author#answer-68310
\makeatletter
\let\runauthor\@author
\makeatother

\lhead{\runauthor}
\chead{Problem Set 6}
\rhead{\today}
\lfoot{}
\cfoot{CS 103: Mathematical Foundations of Computing --- Summer 2024}
\rfoot{\thepage}

\newcommand{\abs}[1]{\lvert #1 \rvert}
\renewcommand{\(}{\left(}
\renewcommand{\)}{\right)}
\newcommand{\floor}[1]{\left\lfloor#1\right\rfloor}
\newcommand{\ceil}[1]{\left\lceil#1\right\rceil}
\newcommand{\pd}[1]{\frac{\partial}{\partial #1}}
\newcommand{\powerset}[1]{\wp\left(#1\right)}
\newcommand{\suchthat}{\ \vert \ }
\newcommand{\naturals}{\mathbb{N}}
\newcommand{\integers}{\mathbb{Z}}
\newcommand{\reals}{\mathbb{R}}
\renewcommand{\qed}{\blacksquare}
\newcommand{\accepts}{\text{ accepts }}
\newcommand{\rejects}{\text{ rejects }}
\newcommand{\loopson}{\text{ loops on }}
\newcommand{\haltson}{\text{ halts on }}
\newcommand{\encoded}[1]{\left\langle#1\right\rangle}
\newcommand{\rlangs}{\mathbf{R}}
\newcommand{\relangs}{\mathbf{RE}}
\newcommand{\corelangs}{\text{co-}\mathbf{RE}}
\newcommand{\plangs}{\mathbf{P}}
\newcommand{\nplangs}{\mathbf{NP}}

\renewcommand{\labelitemii}{$\bullet$}
\renewcommand\qedsymbol{$\blacksquare$}
\newenvironment{prf}{{\bfseries Proof.}}{\qedsymbol}
\renewcommand{\emph}[1]{\textit{\textbf{#1}}}
\newcommand{\annotate}[1]{\textit{\textcolor{blue}{#1}}}
\usepackage{mdframed}
\usepackage{float}

\makeatother

\definecolor{shadecolor}{gray}{0.95}

\theoremstyle{plain}
\newtheorem*{lem}{Lemma}

\theoremstyle{plain}
\newtheorem*{claim}{Claim}

\theoremstyle{definition}
\newtheorem*{answer}{Answer}

\newtheorem{theorem}{Theorem}[section]
\newtheorem*{thm}{Theorem}
\newtheorem{corollary}{Corollary}[theorem]
\newtheorem{lemma}[theorem]{Lemma}

\renewcommand{\headrulewidth}{0.4pt}
\renewcommand{\footrulewidth}{0.4pt}

\setlength{\parindent}{0pt}

\pagestyle{fancy}

\renewcommand{\thefootnote}{\fnsymbol{footnote}}

\usepackage{boxedminipage}
\newenvironment{blank}{\colorbox{shadecolor}{\strut \underline{\ \ \ \ \ }}}

\begin{document}

\maketitle

\begin{center}
  \emph{Due Friday, August 11 at 4:00 pm Pacific}
\end{center}

Some symbols you may want to use here:

\begin{itemize}

\item The empty string is denoted $\varepsilon$.
\item Alphabets are denoted $\Sigma$.
\item The language of an automaton is denoted $\mathscr{L}(D)$.
\item You can say two strings are distinguishable relative to $L$ by writing $x \not\equiv_L y$.
\item You can write a character in monospace \texttt{in text mode} and $\mathtt{in math mode}$.
\item The Optional Fun Problem uses the notation $\mathscr{F}$.

\end{itemize}

\newpage

Problems One and Two are to be answered by editing the appropriate files
(\texttt{RegularExpressions.regexes} and \texttt{Grammars.cfgs}, respectively).
Do not put your answers to Problems One and Two in this file.

\newpage

\section*{Problem Three: Finite Languages}
\begin{shaded}
    Pick an arbitrary finite language $L$. Let $n$ be the number of strings in $L$, that is $|L|=n$. We can view $L$ as containing $n$ strings, that is, $L=\{s_{1},s_{2},\dots,s_{n}\}$. A trivial regular expression for $L$ would then be $s_{1} | s_{2}| \dots | s_{n}$, which ensures that only the $n$ strings in $L$ are accepted.

    \vspace*{4mm}

    Consider the edge case where $|L|=0$. The regular expression for $L$ would then be $\emptyset$, implying that no strings are accepted, as required.
\end{shaded}

\newpage

\section*{Problem Four: Distinguishability}
    i.
    \begin{shaded}
        1. $w=a$

        \vspace*{4mm}

        2. $w=\varepsilon$

        \vspace*{4mm}


        3. $w=\varepsilon$

        \vspace*{4mm}


        4. No such $w$.
    \end{shaded}
    
    ii.
    \begin{shaded}
        1. $aa$ and $aaa$. Both already contain the substring $aa$.

        \vspace*{4mm}

        2. Yes.

        \vspace*{4mm}

        3. $aa$ and $aaa$.

        \vspace*{4mm}

        4. Yes.

        \vspace*{4mm}

        5. Not a distinguishing set. The two strings $abba$ and $abbababba$.

        \vspace*{4mm}

        6. Yes

        \vspace*{4mm}

        7. Not a distinguising set. The two strings are $\varepsilon$ and $b$.
    \end{shaded}
    
    iii.
    \begin{shaded}
        Let $S=\{ab, a, b, baa\}$.

        \vspace*{4mm}

        $ab$ and $a$: $abba\in L$ while $aba\notin L$. $w=ba$ \\
        $ab$ and $b$: $aba\notin L$ while $ba\in L$. $w=a$ \\
        $ab$ and $baa$: $aba\notin L$ while $baaa\in L$. $w=a$ \\

        \vspace*{4mm}

        $a$ and $b$: $aa\notin L$ while $ba\in L$. $w=a$. \\
        $a$ and $baa$: $aa\notin L$ while $baaa\in L$. $w=a$. \\

        \vspace*{4mm}

        $b$ and $baa$: $bb\notin L$ while $baab\in L$. $w=b$.
    \end{shaded}
    
    iv.
    \begin{shaded}
        Theorem: L is not regular.

        \vspace*{4mm}

        Proof: Let $S=\{a^{n}b^{n}|n\in\mathbb{N}\}$. We will prove that $S$ is infinite and that $S$ is a distinguishing set for $L$.

        \vspace*{4mm}

        To see that $S$ is infinite, observe that $S$ contains one string for each natural number.

        \vspace*{4mm}

        To see that $S$ is a distinguishing set for $L$, pick arbitrary strings $a^{n}b^{n},a^{m}b^{m}\in S$ such that $n\neq m$. Let $w=a^{n}b^{n}$. It is trivial that $w\in\Sigma*$. Observe that $a^{n}b^{n}a^{n}b^{n}\notin L$ and $a^{m}b^{m}a^{n}b^{n}\in L$. Therefore, we see that $a^{n}a^{n}\not\equiv_L a^{m}b^{m}$, implying that $S$ is a distinguishing set, as required.

        \vspace*{4mm}

        Since $S$ is infinite and a distinguishing set for $L$, by the Myhill-Nerode Theorem, we see that $L$ is not regular. \qedsymbol
    \end{shaded}
    
\newpage

\section*{Problem Five: Brzozowksi's Theorem}
    i.
    \begin{shaded}
        1. $aab$
        
        \vspace*{4mm}

        2. $bbbaaa$

        \vspace*{4mm}

        3. $bba$
    \end{shaded}
    
    ii.
    \begin{shaded}
        Proof: We want to show that L is not regular. It is given that $S$ is infinite, hence, we only need to show that $S$ is a distinguishing set for $L$. To do so, pick arbitrary strings $x,y\in S$ such that $x\neq y$. Since $x\neq y$, we know that $\partial_x L\ne \partial_y L$. Since $\partial_x L\ne \partial_y L$, we know that, without loss of generality, there exists a string $w$ such that $w\in\partial_x L$ and $w\notin\partial_y L$. This implies that $xw\in L$ and $yw\notin L$, which tells us that $x\not\equiv_L y$. Hence, we see that $S$ is a distinguishing set for $L$ as required. 

        \vspace*{4mm}

        Since $S$ is infinite and a distinguishing set for $L$, by the Myhill-Nerode Theorem, we see that $L$ is not regular. \qedsymbol
    \end{shaded}

\newpage

\section*{Optional Fun Problem: Birget's Theorem}
\begin{shaded}
Write your answer to the Optional Fun Problem here.
\end{shaded}

\end{document}
